\documentclass[10pt]{article}

\usepackage{geo-top}

\begin{document}

\lhead{Frederick Robinson} \rhead{Fall 2002}

\question{Suppose $P(x,y,z)$, $Q(x,y,q)$, and $R(x,y,z)$ are \smooth functions on $\R^3$ which
  vanish identically if $|x| \geq 5$, $|y| \geq 5$, or $|z| \geq 5$. Prove that the volume
  integral
$$\int_{-6}^6 \int_{-6}^6 \int_{-6}^6 d(P dy \wedge dz + Q dx \wedge dz + R dx \wedge
dy) =0$$ (Do this directly, not by quoting Stokes' Theorem: this is a special case of the proof of
Stokes' Theorem!)  }

\question{Suppose that $V = P(x,y,z) \frac{\del}{\del x} + Q(x,y,z) \frac{\del}{\del y} + R(x,y,z)
  \frac{\del}{\del z} $ is a \smooth vector field on $\R^{3}$ with $V \not= \vec{0}$ at the
  origin. Find a necessary and sufficient condition for there to exist a \smooth function
  $\lambda(x,y,z) $ in some neighborhood of the origin such that $\lambda V$ is the gradient of a
  \smooth function on the neighborhood.  }

\question{Let $T_t : \R^3 \to \R^3$ be the right-hand rule rotation around the positive $z$-axis by
  $t$ degrees and $S_s : \R^3 \to \R^3 $ be the right-hand rule rotation around the positive
  $x$-axis by $t$ degrees.}

\subquestion{Find the infinitesimal generators of the flows $T_t$ and $S_t$, i.e., the vector fields
$X$ and $Y$, respectively, on $\R^3$ whose flows are $\{T_t\} $ and $\{ S_t\}$.}

\subquestion{Compute the commutator $$T_{-t}\circ S_{-t}\circ T_t\circ S_t .$$}
\label{comm}

\subquestion{Compare the result of \ref{comm} (lowest order non-identically zero term) with the Lie bracket
$[X,Y]$.}

\question{Take as given that a \smooth $2$-form $\omega$ on $S^2$ is of the form $d \theta$ for
  some \smooth $1$-form $\theta$ if and only if $\int_{X^2} \omega =0$. Use this to show that
  every \smooth $2$-form $\Omega$ on \RP{2} has the form $d \Lambda $ for some
  \smooth $1$-form $\Lambda$. (Do not just quote DeRham's Theorem here.)}

page

\nextquestion

\subquestion{Suppose $F: S^1 \to \R^3$ is a \smooth function such that $dF$ is nowhere
  zero (on $S^1$). Prove that there is a two-dimensional subspace $P$ of $\R ^3$ such that
  $\pi_P \circ F : S^1 \to \R^3$ has nowhere vanishing differential, where $\pi_P =$
  orthogonal projection on $P$.}

\subquestion{Show by example a picture with explanation is all right) that there is such an
  $F$ that is also 1 to 1 (injective) but is such that, for all $P$, $\pi_P \circ F$ fails to be
  injective.}

\subquestion{Show that if $F: S^1 \to \R^4$ is \smooth and injective then there is a
  three-dimensional subspace $H$ of $\R^4$ such that $\pi_H \circ F $ is injective, where
  $\pi_H =$ orthogonal projection on $H$.}

\nextquestion

\subquestion{Suppose $F:S^n \to S^n$ is fixed-point free (i.e. for all $ p \in S^n$, $p \neq
  F(p)$). Show that $F$ is homotopic to the antipodal map $p \to -p$, $p \in S^n$.}
\label{antip}

\subquestion{Use part \ref{antip} to show that every vector field on (tangent to) $S^{2n}$,
  $n=1,2,3,\dots$ vanishes somewhere on $S^{2n}$ (i.e. has a zero).}

\nextquestion

\subquestion{Discuss carefully how to obtain the long exact sequence in homology from a short exact
  sequence of chain complexes. (Include definitions of the maps in the long exact sequence.)}

\subquestion{If the short exact sequence is 
$$0 \to C_1 \to C_2 \to C_3 \to 0,$$
prove exactness of the long exact sequence at $H_k(C_3)$ [in $\cdots H_k (C_2) \to H_k(C_3) \to
H_{k-1}(C_1) \cdots$]}

\nextquestion

\subquestion{Suppose $F: T^2 \to T^2$ (where $T^2 = S^1 \times S^1$ is a continuous function such
  that $F(p ) =p$ for some $p \in T^2$ and
$$F_* : \pi_1(T^2, p) \to \pi_1(T^2, p) $$
is the identity map. Is $F$ necessarily homotopic to the identity map from $T^2$ to itself?  }

\subquestion{Is a \smooth map $F: T^2 \to T^2$ of degree 1 necessarily homotopic to the identity
  map of $T^2$ to itself? Explain / prove your answer.}

\nextquestion

\subquestion{Discuss the (a) representation of \CP{n} as a simplicial complex.}

\subquestion{Use part (a) to find the homology of \CP{n}: prove carefully that your
  calculation is correct.}

\nextquestion

\subquestion{Let $X =$ the space obtained by attaching two discs to $S^1$, the first disc being
  attached by $S^1 = \del D_1 \to S^1$ being the 7 times around (counterclockwise) map, e.g.,
  $z\mapsto z^7$, $|z|=1$, $z \in \C $ and the second being attached by $S^1 = \del D_2 \to
  S^1$ being the 5 times around map $z \to z^5$. Find the homology of $X$.}


\subquestion{Can $X$ be made a \smooth manifold? Why or why not?}





\end{document}

\documentclass[10pt]{article}

\usepackage{geo-top}

\begin{document}

\lhead{Frederick Robinson} \rhead{Fall 2010}

\section{Let $M$ be a connected smooth manifold. Show that for any two non-zero tangent vectors
  $v_1$ at point $x_1$ and $v_2$ at a point $x_2$, there is a diffeomorphism $\phi : M \to M$ such
  that $\phi(x_1) = x_2$ and $d\phi(v_1) = v_2$.}

\section{Let $X$ and $Y$ be submanifolds of $\R^n$. Prove that for almost ever $a \in \R^n$, the
  translate $X + a$ intersects $Y$ transversely.}

\section{Let $\mat{n}{n}{\R} \simeq \R^{n^2}$ be the space of $n \times n$ matrices with real
  coefficients.}
\subsection{Show that $$ \SL{n}{\R} = \{ A \in \mat{n}{n}{\R} \mid \det(A) = 1 \}$$ is a smooth
  submanifold of \mat{n}{n}{\R}.}
\subsection{Identify the tangent space to \SL{n}{\R} at the identity matrix $I_n$.}
\subsection{Show that \SL{n}{\R} has trivial Euler characteristic.}

\advsection
\subsection{Let $f_i : M \to N$, $i = 0,1$, be two smooth maps between smooth manifolds $M$ and $N$,
  and $f_i^* : \Omega^*(N) \to \Omega^*(M)$, $i = 0,1$, be the induced chain maps between the
  respective de Rham complexes. Define the notion of chain homotopy between $f_0^*$ and
  $f_1^*$. Here the co-boundary operators on the de Rham complexes are the exterior derivatives.}
\subsection{Let $X$ be a smooth vector field on a compact smooth manifold $M$, and let $\phi_t: M
  \to M$ be the flow generated by $X$ at time $t$, i.e. the solution of the differential equation
  $\frac{d \phi_t}{dt}(x) = X(\phi_t(x))$ with initial condition $\phi_0(x) =x$. Find an explicit
  chain homotopy between the chain maps $\phi_0^*$ and $\phi_1^*$, where $\phi_i^*$, $i = 0,1$, are
  the induced chain maps from $\Omega^*(M)$ to itself. \\ \hint{Use the formula that for any
    differential form $\omega$ and vector field $X$, the Lie derivative $\mathcal{L}_X \omega = d
    \circ i_X\omega + i_X \circ d \omega$. Here $i_X$ is the contraction with respect to $X$.} }

split

\section{Let $\omega = dx_1 \wedge dx_2 + dx_3 \wedge dx_4 + \cdots + dx_{2n-1}\wedge dx_{2n}$ be a
  2-form on $\R^{2n}$, where $(x_1, x_2, \dots, x_{2n})$ are the standard coordinates on
  $\R^{2n}$. Define an $S^1$-action on $\R^{2n}$ as follows: for each $t \in S^1$, define $g_t :
  \R^{2n} \to \R^{2n}$ by considering $\R^{2n}$ as the direct sum of $n$ copies of $\R^2$ and
  rotating each $\R^2$ summand an angle $t$. Let $X$ be the vector field on $\R^{2n}$ defined by
  $X(x) = \frac{dg_t(x)}{dt} |_{t=0}$ for any $x \in \R^{2n}$. }
\subsection{Find the Lie derivative $\mathcal{L}_X \omega$ and a function $f$ on $\R^{2n}$ such that
  $df = i_X \omega$. }
\label{5a}
\subsection{The $S^1$-action above induces an action on $S^{2n-1}$. Let $\mathbb{P}^{n-1}$ be the
  quotient space of $S^{2n-1}$ by this $S^1$-action. Show that the quotient space $\mathbb{P}^{n-1}$
  has a natural smooth structure and that the tangent space of $\mathbb{P}^{n-1}$ at any point
  $\underbar{x}$ can be identified with the quotient of the tangent space $T_xS^{2n-1}$ by the line
  spanned by $X(x)$, for any $x \in \underbar{x}$. Here $\underbar{x}$ is the orbit of $x$ under the
  $S^1$-action.}
\label{5b}
\subsection{Show that $\omega$ descends to a well-defined 2-form on the quotient space
  $\mathbb{P}^{n-1}$ and that the 2-form so defined is closed.}
\label{5c}
\subsection{Is the closed form in (\ref{5c}) exact?}
\section*{ \hint{For (\ref{5c}) and (\ref{5d}) use
    (\ref{5a}) and (\ref{5b})} }
\label{5d}

\section{Suppose that $f: S^n \to S^n$ is a smooth map of degree not equal to $(-1)^{n+1}$. Show
  that $f$ has a fixed point.}

\advsection
\subsection{Let $G$ be a finitely presented group. Show that there is a topological space $X$ with
  fundamental group $\pi_1(X) \cong G$.}
\subsection{Give an example of $X$ in the case $G = \Z * \Z$, the free group on two generators.}
\label{7b}
\subsection{How many connected, 2-sheeted covering spaces does the space $X$ from (\ref{7b}) have?}

\section{Let $G$ be a connected topological group. Show that $\pi_1(G)$ is a commutative group.}

\section{Show that if $\R^m$ and $\R^n$ are homeomorphic, them $m = n$.}

\section{Let $N_g$ be the nonorientable surface of genus $g$, that is, the connected sum of $g$
  copies of \RP{3}. Calculate the fundamental group and homology groups of $N_g$.}


\end{document}

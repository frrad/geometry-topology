\documentclass[10pt]{article}

\usepackage{geo-top}

\begin{document}

\lhead{Frederick Robinson} \rhead{Spring 2012}

\question{Explain in detail from the viewpoint of transversality theory, why the sum of the indices
  of a vector field with isolated zeroes on a compact orientable manifold $M$ is independent of what
  vector field we choose.}
\label{transv}

\question{Call the index sum in problem \ref{transv} the Euler characteristic $\chi(M)$. Explain why
  the Euler characteristic of a genus $g$ surface (2-sphere with $g$ handles attached) is
  $2-2g$. [Do this explicitly: do \emph{not} appeal to the theorem that the Euler characteristic in
  the vector field sense indicated is computable from homological information. That comes next!]}

\question{Suppose that $M$ is a triangulated compact orientable manifold, i.e., a manifold $M$
  represented as a finite simplicial complex.}

\subquestion{Show that the alternating sum of the Betti numbers $b_0 - b_1 + b_2- \cdots$ (where $b_k
  = \rank$ of the $k$th homology group with real coefficients) is equal to the alternating sum
  $(\mbox{number of vertices}) - (\mbox{number of faces}) + (\mbox{number of
  }2-\mbox{simplices})-\cdots$}
\label{equiv}

\subquestion{Show that there is a vector field with the sum of its indices equal to the number
  described in part \ref{equiv}. [You do not need to worry about smoothness of the vector field --
  just describe how to build it. In part \ref{equiv}, the result should follow from some dimension
  counting.]}

\question{Suppose $V$ is a smooth ($C^\infty$) vector field on $\R^3$ that is nonzero at
  $(0,0,0)$. The vector field is said to be gradient-like at $(0,0,0)$ if there is a neighborhood of
  $(0,0,0)$ and a nowhere zero smooth function $\lambda(x,y,z)$ on that neighborhood such that
  $\lambda V$ is the gradient of some smooth function in some (possibly smaller) neighborhood of
  $(0,0,0)$.}

\subquestion{Write $V = (P,Q,R)$. Show by example that there are functions $P,Q,R$ for which $V$ is
  not gradient-like in a neighborhood of $(0,0,0)$. \hint{the orthogonal complement of $V$ taken at
    each point would have to be an integrable $2$-plane field}}

\subquestion{Derive a general differential condition on $(P,Q,R)$ which is necessary and sufficient
  for $V$ to be gradient-like in a neighborhood of $(0,0,0)$.}

\nextquestion

\subquestion{Define carefully the ``boundary map'' which defines the $H_n$ to $H_{n-1}$ mapping that
  arises in the long exact sequences arising from a short exact sequence of chain complexes.}

\subquestion{Prove that the kernel of the boundary map is equal to the image of the map into the $H_n$.}

answer

\question{Compute the homology of the real projective space \RP{n} for each $n > 1$.}

\nextquestion

\subquestion{Define complex projective space \CP{n} $(n = 1,2,3, \dots)$}

\subquestion{Show that \CP{n} is compact for all $n$.}

\subquestion{Show that \CP{n} has a cell decomposition with one cell in each dimension
  $0,2,4,\dots, 2n$ and no other cells. Include a careful description of the attaching maps.}

\question{Suppose a compact (real) manifold $M$ has a (finite) cell decomposition with only even
  dimensional cells. Is $M$ necessarily orientable? Justify your answer.}

\question{Suppose that a finite group $\Gamma$ acts smoothly on a compact manifold $M$ and that the
  action is free, i.e. $\gamma(x) = x$ for some $x$ in $M$ if and only if $\gamma =$ the identity of
  the group $\Gamma$.}

\subquestion{Show that $M  / \Gamma$ is a manifold (i.e., can be made a manifold in a natural way)}

\subquestion{Show that $M \to M / \Gamma$ is a covering space.}

\subquestion{If the $k$th de Rham cohomology of $M$ is 0, some particular $k > 0$, then is the $k$th
  de Rham cohomology of $M / \Gamma$ necessarily 0? Prove your answer.}

\question{Let $M = \RP{2} \times \RP{2}$ where \RP{2} is a real projective
  2-space). In a product manifold like that, homology elements can arise by taking in effect the
  product of a cycle in one factor with a cycle in the other factor. Show that in the case of this
  particular $M$, there is an element in the 3-homology with \Z coefficients that does not
  arise in this way by exhibiting such an element explicitly, e.g. in terms of a cell
  decomposition.}






\end{document}

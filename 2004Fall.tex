\documentclass[10pt]{article}

\makeatletter \renewcommand\section{\@startsection{section}{1}{\z@}%
  {-3.5ex \@plus -1ex \@minus -.2ex}%
  {2.3ex \@plus.2ex}%
  {\normalfont\large\bfseries}} \makeatother

\addtolength{\oddsidemargin}{-.875in} \addtolength{\evensidemargin}{-.875in}
\addtolength{\textwidth}{1.75in} \addtolength{\topmargin}{-.875in}
\addtolength{\textheight}{1.75in}

\usepackage{amsmath, amssymb, amsthm, fancyhdr, graphicx, titlesec, titling}
\usepackage[all]{xy}

\setlength{\droptitle}{-6em} \posttitle{\par\end{center}\vspace{-4.8em}}

\newcommand{\C}{\ensuremath{\mathbb{C}}}
\newcommand{\CP}{\ensuremath{\mathbb{CP}}}
\newcommand{\R}{\ensuremath{\mathbb{R}}}
\newcommand{\RP}{\ensuremath{\mathbb{RP}}}
\newcommand{\Z}{\ensuremath{\mathbb{Z}}}

\newcommand{\del}{\ensuremath{\partial}}
\newcommand{\rank}{\ensuremath{\mathrm{rank}}}
\DeclareMathOperator{\dev}{div}

\newcommand{\advsection}{\addtocounter{section}{1} \setcounter{subsection}{0}}
\newcommand{\st}{s.t. }
\newcommand{\hint}[1]{(Hint: #1)}
\newcommand{\note}[1]{(Note: #1)}

\pagestyle{fancyplain} \renewcommand{\headrulewidth}{0pt}

\begin{document}

\lhead{Frederick Robinson} \rhead{Fall 2004}

\section{Let $M$ be a connected smooth manifold. Construct the orientation cover $M_0$.}

\subsection{Show that $M_0$ is a smooth manifold.}

\subsection{Show that $M_0$ is a $2:1$ covering of $M$.}

\subsection{Show that $M$ is orientable iff $M_0$ is the union of two disconnected components.}

\section{Let $\omega$ be a smooth nowhere vanishing 1-form on a smooth connected manifold $M$.}

\subsection{Show that $\ker \omega$ is a smooth codimension 1 distribution on $M$.}

\subsection{Show that $\ker \omega$ is integrable iff $d\omega$ vanishes on $\ker \omega$.}

\subsection{Find a codimension 1 distribution on $\mathbb{R}^3$ which is not integrable.}

\section{Show that $S^1 \times S^n$ is parallelizable, i.e., one can find $(n+1)$ vector fields that
  are everywhere linearly independent. ($S^k \subset \mathbb{R}^{k+1}$ is the unit sphere.)}

\section{Let $\omega = \frac{-y dx + x dy}{(x^2 + y^2)^\alpha} $ and consider $\int_\gamma \omega$,
  where $\gamma : S^1 \to \mathbb{R}^2 \setminus \left\{ 0 \right\} $.}

\subsection{For which $\alpha$ is $\int_{\gamma_0} \omega = \int_{\gamma_1} \omega$, whenever
  $\gamma_0$ and $\gamma_1$ are smoothly homotopic, i.e., there exists $F:S^1 \times [0,1] \to
  \mathbb{R}^2\setminus \left\{ 0 \right\}$ such that $\gamma_0(t) = F(t,0)$, $\gamma_1(t) = F(t,1)$?}
  \label{cre}

\subsection{What are the possible values for $\int_\gamma \omega$ when $\alpha$ is chosen as in part
  \ref{cre}?}

\section{Show that a closed (compact without boundary) $n$-manifold cannot be immersed in
  $\mathbb{R}^n$.}

\section{Let $\mathbb{C}^*$ be the set of all nonzero complex numbers with the induced topology from
  $\mathbb{C}$. It is a topological group with respect to the usual multiplication. Let $f$ be a
  continuous homomorphism from $\mathbb{C}^*$ to itself.}

\subsection{Find all possible $f|_{S^1}$, where $S^1 = \left\{ z \mid |z| = 1, z \in \mathbb{C}^*
  \right\} $.}

\subsection{Classify such $f|_{S^1}$ up to homotopy.}

\section{Let $X_1 = S^1 \vee_{x_1 = x_2} S^2$ be the space obtained from the disjoint union of the
  circle $S^1$ and the $S^2$ by identifying a point $x_1 \in S^1$ with a point $x_2 \in S^2$. Define
  $X_2 = S^1 \vee_{x_1 = x_2} S^1$ similarly.}

\subsection{Find $\pi_1(X_1)$ and $\pi_1(X_2)$.}

\subsection{Find their universal coverings.}

\section{Let $f: S^2 \to T^2$ be a continuous map from 2-sphere to 2-torus $T^2$. What is the
  induced map 
  $$f_* : H_*(S^2) \to H_*(T^2)$$
  on the homology groups?}

\section{Let $X$ be a topological space, and define $S(X)$ to be the quotient space of $X\times I$
  by contracting $X \times \left\{ 0 \right\} $ to a point and $X \times \left\{ 1 \right\} $ to
  another point. Here $I = [0,1]$. What is the relationship between $H_*(S(X))$ and $H_*(X)$?}

\section{Let $K$ be a finite simplicial complex and $K^n$ be the subcomplex consisting of all
  simplices in $K$ of dimension less than or equal to $n$. Denote the underlying topological spaces
  of $K$ and $K^n$ by $|K|$ and $|K^n|$.}

\subsection{What is the relative singular homology $H_*(|K|, |K^{n-1}|)$?}
\label{former}

\subsection{Write down the long exact sequence for the triple $(|K^n|, |K^{n-1}|, |K^{n-2}|)$, i.e.,
  the long exact sequence relating the singular homology groups $H_*(|K^n|, |K^{n-1}|)$,
  $H_*(|K^{n-1}|, |K^{n-2}|)$ and  $H_*(|K^n|, |K^{n-2}|)$.}
\label{latter}

\subsection{Use \ref{former}  and \ref{latter} to show that singular homology of $|K|$ is the same as the simplicial homology
  of $|K|$. \hint{Identify the connecting boundary map in \ref{latter}}}



\end{document}

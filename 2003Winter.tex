\documentclass[10pt]{article}

\usepackage{geo-top}

\begin{document}

\lhead{Frederick Robinson} \rhead{Winter 2003}

\section{Let $M$ be a smooth compact manifold of dimension $n$. Show that there is no immersion of
  $M$ into $\mathbb{R} ^n$.}

\section{The $n$-dimensional torus $T^n$ is defined to be $\mathbb{R} ^n / \mathbb{Z} ^n$, i.e. for
  any $x$ and $y$ in $\mathbb{R}^n$, $x \sim y$ iff $x-y \in \mathbb{Z}^n$. Let $\alpha$ and $\beta$
  be two such functions on $\mathbb{R}^n$ such that (i) $\alpha(x) = \alpha(y)$ and $\beta(x) =
  \beta(y)$ iff $x-y \in \mathbb{Z}^n$ and (ii) $\alpha / \beta$ is an irrational constant. Then 
  $$v = \alpha(x) \frac{\partial }{\partial x^1} + \beta(x) \frac{\partial }{\partial x^2} $$
  is a vector field on $\mathbb{R}^n$ descending to $T^n$, where 
  $$\left\{ \frac{\partial }{\partial x^1}, \frac{\partial }{\partial x^2}, \dots, \frac{\partial
    }{\partial x^n} \right\}$$
  are coordinate vector fields. Find all functions $f$ on $T^n$ such that $v f =0$.}

\section{Let $M$ and $N$ be smooth compact connected manifolds and $f: M \to N$ be a smooth map such
  that, for any point $m \in M$, $\rank(df_m) = \dim(N)$. Show that (i) for any $n \in N$,
  $f^{-1}(n)$ is a submanifold of $M$ and (ii) for any $n_1$ and $n_2$ in $N$, the submanifolds
  $f^{-1}(n_1)$ and $f^{-1}(n_2)$ of $M$ are diffeomorphic to each other.}

\section{Let
  $$ \tilde{\theta} = \frac{1}{2} \left\{ (x^2 dx^1 - x^1 dx^2) + (x^4 dx^3 - x^3 dx^4) + \cdots + (
      x^{2n} dx^{2n-1} - x^{2n-1} dx^{2n}) \right\} $$ 
  be a 1-form on $\mathbb{R}^{2n}$ and $\theta$ be its restriction to the unit sphere 
  $$S^{2n-1} = \left\{ x = (x^1, \dots, x^{2n}) \mid (x^1)^2+ \cdots + (x^{2n})^2 =1  \right\}. $$
  The kernel $K$ of $\theta$ is a distribution on $S^{2n-1}$:
  $$K = \left\{ v \mid v \in TS^{2n-1}, \theta(v) = 0 \right\}. $$
  Decide whether or not $K$ is integrable.}

\section{Let $T^{2m} = \mathbb{R}^{2n} / \mathbb{Z}^{2n}$ be a torus of dimension $2n$. Consider the
  2-form
  $$\omega = dx^1 \wedge dx^{n+1} + dx^2 \wedge dx^{n+2} + \cdots + dx^n \wedge dx^{2n}$$
  defined on $R^{2n}$ descending to $T^{2n}$.}

\subsection{Show that $\omega$ is closed by not exact on $T^{2n}$.}

\subsection{Let $i : T^n \to T^{2n}$ be the subtorus defined by the equation
  $$x^{n+1} =x^{n+2} = \cdots =x^{2n}=0.$$
  What is $i^*\omega$?}

\subsection{Let $\Sigma = S^2 \setminus \left\{ \cup^m_{i=1} D_i \right\}$, where $D_i$,
  $i=1,\dots,m$ are $m$ open discs in $S^2$ which disjoint closures. Show that
  $$ \int_\Sigma f^*_1 \omega = \int_\Sigma f^*_2 \omega$$
  if $f_1, f_2 : (\Sigma, \partial \Sigma) \to (T^{2n}, T^n)$ are homotopic to each other, where
  $\partial \Sigma$ is the boundary of $\Sigma$.}

\section{Let $X$ be a path connected space and let $x_0, x_1 \in X$. Prove carefully that
  $\pi_1(X,x_0)$ is isomorphic to $\pi_1(X,x_1)$.}


\advsection{}

\subsection{Define what is meant by a ``chain homotopy'' $P$ between chain maps $f_\#, g_\#: C \to
  D$ and prove that chain homotopic chain maps induce the same homomorphism of homology.}

\subsection{Let $X$ and $Y$ be spaces and let $F: X \times I \to Y$ be a homotopy between maps $f$
  and $g$. Define a chain homotopy $P$ between the induced chain maps $f_\#, g_\#: C(X) \to C(Y)$ of
  singular chains.}

\subsection{Verify that $P$ satisfies the definition of a chain homotopy \emph{only} for the
  restriction of $P$ to $C_1(X)$.}

\section{Let $X$ be a locally contractible space and $H$ a subgroup of $\pi_1(X,x_0)$. Describe
  carefully how to construct a topological space $X_H$ and a map $p:X_H \to X$ such that
  $p_*(\pi_1(X_H,\tilde{x}_0))=H$ and show that it has the required property of
  $p_*$. \note{Although $X_H$ will be a covering space, you don't have to verify this unless you
    want to use some general properties of covering spaces.}}

\section{Prove that the real even-dimensional projective spaces $\mathbb{R}P^{2n}$ have the fixed
  point property, that is, for every map $f: \mathbb{R}P^{2n} \to \mathbb{R}P^{2n}$ there is a
  solution to $f(x) =x$. \hint{Consider the maps on the covering space $S^{2n}$.}}

\section{Use the Mayer-Vietoris sequence to calculate the homology of $S^1 \times S^2$. You may
  assume the homology calculations for $S^1$, $S^1 \times S^1$ and $S^2$.}

\end{document}

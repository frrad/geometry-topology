\documentclass[10pt]{article}

\usepackage{geo-top}

\begin{document}

\lhead{Frederick Robinson} \rhead{Fall 2003}

\question{Explain carefully how the classical ``divergence theorem''
$$\iint\limits_S \vec{V} \cdot \vec{n} d(\mathrm{area}) = \iiint\limits_V \dev \vec{V} d(\mathrm{volume})$$
($V$ a bounded volume in $\R^3, S= $ boundary of $V$) follows from Stokes' Theorem for
differential forms.}

\question{Without using deRham's Theorem, prove:}

\subquestion{every closed 1-form on $S^2$ is exact.}

\subquestion{a two-form $\Omega$ is exact on $S^2$ if and only if 
$$\int_{S^2}\Omega=0.$$}

\question{Show that the set of all lines in $\R^2$ has a natural structure as a
  differentiable manifold. What (already familiar) manifold is it?}

\question{Show that $S^3$ is the union of two solid tori $(S^1 \times 2$-disc) with an embedded torus
  $(S^1 \times S^1)$ as their common boundary. \hint{Express $\R^3$ with a solid torus
    removed as a union of circles and a single straight line and then add a point at infinity.}}

\question{Suppose $M$ is a compact manifold (with empty boundary).}

\subquestion{Prove that, if $f: M \to \R$ is a \smooth function, then $f$ has at least two
critical points.}

\subquestion{A \smooth function on $S^1 \times S^1$ cannot have only two critical points. Prove
  this (e.g.) by deforming a homotopically nontrivial $S'$ along the gradient flow of $f: M \to \R$. }

\nextquestion

\subquestion{Prove carefully that a group of homeomorphisms of $S^{2n}$, each of which has no fixed
  points (unless it is the identity map) contains at most two elements.}

\subquestion{Give a counterexample for some $S^{2n+1}$, $n \geq 1$.}

\question{Find the homology groups with $\Z$ coefficients, of \RP{n}, $n=2,3,4,
  \dots$ by some systematic rigorous method.}

\question{Find the homology and the fundamental group of $S^1 \times S^1$ with two points removed.}

\question{Prove that if a compact (empty boundary) manifold $X$ has $S^{2n+1}$, $n \geq 1$, as a
  covering space, then $X$ is orientable.}

\question{Suppose $M$ is a compact manifold (empty boundary). Prove that
$$H_n(M,\Z ) \simeq \Z.$$
(You may assume $M$ is triangulated.)}

\end{document}

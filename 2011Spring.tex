\documentclass[10pt]{article}

\usepackage{geo-top}

\begin{document}

\lhead{Frederick Robinson} \rhead{Spring 2011}

\section{Show that if $V$ is a smooth vector field on a (smooth) manifold of dimension $n$ and if
  $V(p)$ is nonzero for some point $p$, then there is a coordinate system defined on a neighborhood
  of $p$, say $(x_1, \dots, x_n)$, such that on a neighborhood of $p$, $V =$ the $x_1$ coordinate
  vector field.}

\advsection{}

\subsection{Demonstrate the formula $L_x = di_x + i_x d$, where $L$ is the Lie derivative and $i$ is
  the interior product.}

\subsection{Use this formula to show that a vector-field $X$ on $\mathbb{R}^3$ has a flow (locally)
  that preserves volume if and only if the divergence of $X$ is everywhere 0. [Here divergence is
  the classical operator that takes a vector field with components $f,g,h$ to the function $f_x +
  g_y + h_z$, in the usual partial derivative notation $f_x = x$-partial of $f$, etc.]}

\advsection{}

\subsection{Explain some systematic reason why there is a closed 2-form on $\mathbb{R}^3 \setminus
  \left\{ (0,0,0) \right\} $ (Euclidean 3-space with one point removed) that is not exact. You man
  do this by exhibiting such a form explicitly and checking that it is closed but not exact or you
  may argue using theorems that such a form must exist.}
\label{above}

\subsection{With $\varphi$ such a form (as in part \ref{above}), discuss why, for any smooth map of
  $S^2$ to itself, such a number
  $$\left(\int_{S_2} F^* \varphi\right) / \left(\int_{S_2}\varphi  \right)$$
  is the degree of $f$. [Note that this includes explaining why the denominator integral cannot be 0.].}

\section{Show without using de Rham's Theorem (but you may use the Poincare Lemma without proof),
  that a 2-form $\varphi$ on the 2-sphere $S^2$ that has integral 0 is exact, i.e., equal to $d
  \omega$ for some 1-form $\omega$ on $S^2$.}

\section{Suppose that $V: U \to S^2$ is a smooth map, considered as a vector field of unit vectors,
  where $U = \mathbb{R}^3 $ with a finite number of points $p_1, \dots, p_n$ removed, all these
  points lying strictly inside the unit sphere $S^2$. \\\\ Explain carefully, from basic facts about
  critical values and critical points and the like, why the degree of $V|_{S^2} : S^2 \to S^2$ is
  equal to the sum of the indices of the vector field $V$ at the points $p_1, \dots, p_n$.}

\advsection{}

\subsection{Explain what a short exact sequence of chain complexes is.}

\subsection{Describe how a short exact sequence of chain complexes gives rise to a long exact
  sequence in homology. Include how the connecting homomorphism (where the dimension changes)
  arises. You do not need to prove exactness of this sequence.}

\advsection{}

\subsection{Define complex projective space $\mathbb{C} P^n$, $n = 1,2,3, \dots$}

\subsection{Compute the homology and cohomology of $\mathbb{C} P^n$, $\mathbb{Z}$ coefficients. (Any
  method is allowed. Cell complexes are particularly simple to use. Be sure to explain what the
  attaching maps are if you adopt this approach).}

\advsection{}

\subsection{Find the $\mathbb{Z}$-coefficient homology of $\mathbb{R} P^2$ by any systematic
  method.}

\subsection{Explain explicitly (not using the K\"unneth Theorem) how a nonzero element of the
  3-homology with $\mathbb{Z}$-coefficients of $\mathbb{R} P^2 \times \mathbb{R} P^2$ arises.}

\advsection{}

\subsection{State the Lefschetz Fixed Point Theorem.}

\subsection{Show that the Lefschetz number of any map from $\mathbb{C} P^{2n}$ to itself is nonzero
  and hence that every map from $\mathbb{C} P^{2n}$ to itself has a fixed point. \hint{The action of
    the map on cohomology with $\mathbb{Z}$ coefficients is determined by what happens to the 2nd
    cohomology since the whole cohomology ring is generated by the 2nd cohomology.}}

\section{Compute explicitly the simplicial homology, $\mathbb{Z}$ coefficients, of the surface of a
  tetrahedron, thus obtaining the homology of the 2-sphere.}

\end{document}
